\documentclass[xcolor=pdftex,dvipsnames,table]{beamer}
\usepackage{subfigure}
\usepackage{amsbsy}
\usepackage{tikz}
\usetikzlibrary{arrows}
\usepackage{amsmath,graphicx,dsfont,color}
\usepackage{amsfonts}
\usepackage{amssymb}
\usepackage{array}

\bibliographystyle{apalike}

\setbeamertemplate{bibliography item}{\insertbiblabel}
\setbeamertemplate{bibliography entry title}{}
\setbeamertemplate{bibliography entry location}{}
\setbeamertemplate{bibliography entry note}{}

%Definitiona

\newcommand{\x}{\mathbf{x}}
\newcommand{\X}{\mathbf{X}}
\newcommand{\W}{\mathbf{W}} %Weight
\newcommand{\bais}{\mathbf{b}}%Bais
\newcommand{\act}{\texttt{g}}%Activation
\newcommand{\loss}{L}
\newcommand{\pdata}{\hat{p}_{\texttt{data}}}
\newcommand{\nsize}{n}
\newcommand{\param}{\boldsymbol{\theta}}
\newcommand{\featmap}{\boldsymbol{\phi}}
\newcommand{\EV}{\mathbb{E}}









\title{Deep Learning for Image Analysisg}
\author{Ph.D. Santiago VELASCO-FORERO}
\date{ \'Ecole des Mines de Paris \\ MINES Paris-Tech}
%To include LOGO?
%\logo{\includegraphics[width=.1\columnwidth]{MinesLogo}}
\useinnertheme{rounded}
\usecolortheme{rose}

\begin{document}
\begin{frame}
\titlepage
\end{frame}


\begin{frame}{Negative conditional log-likelihood}
The negative conditional log-likelihood of training data can be written as:
\begin{equation}
J(\param)=\EV_{\X,y \sim \pdata} \loss(\mathbf{x},y,\param) = \frac{1}{\nsize} \sum_{i=1}^{\nsize} \loss(\x^{(i)},y^{(i)},\param),
\end{equation}
where $\loss$ is the per-example loss $\loss(\x,y,\param)=-\log p(y | \x ; \param)$
\end{frame}

\begin{frame}{Architecture Design}
\begin{itemize}
\item The word \emph{architecture} refers to the overall structure of the network.
\item Most neural networks are organized into groups of units called \emph{layers}
\item Most neural networks architectures arranges these layers in a \emph{chain structure}, with each layer being a function of the layer that preceded it.
\end{itemize}
In this structure, the first layer is given by:
\begin{equation}
h^{(1)}=\act^{(1)} (\W^{(1)T}\x + \bais^{(1)})
\end{equation}
where $\act$ denotes the \emph{activation function}.
Accordingly, the second layer is given by:
\begin{equation}
h^{(2)}=\act^{(2)} (\W^{(2)T}\x + \bais^{(2)})
\end{equation}
and so on.
\end{frame}

\end{document}