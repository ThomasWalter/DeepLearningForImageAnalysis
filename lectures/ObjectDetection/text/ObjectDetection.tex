%\documentclass[handout,xcolor=pdftex,dvipsnames,table,mathserif]{beamer}
\documentclass[xcolor=pdftex,dvipsnames,table]{beamer}
\usepackage{subfigure}
\usepackage{amsbsy}
\usepackage{tikz}
\usetikzlibrary{arrows}
\usepackage{amsmath,graphicx,dsfont,color}
\usepackage{amsfonts}
\usepackage{amssymb}
\usepackage{array}

\bibliographystyle{apalike}

\setbeamertemplate{bibliography item}{\insertbiblabel}
\setbeamertemplate{bibliography entry title}{}
\setbeamertemplate{bibliography entry location}{}
\setbeamertemplate{bibliography entry note}{}

\DeclareMathOperator*{\argmin}{arg\,min}
\DeclareMathOperator*{\argmax}{arg\,max}
%Definitiona

\newcommand{\x}{\mathbf{x}}
\newcommand{\X}{\mathbf{X}}
\newcommand{\W}{\mathbf{W}} %Weight
\newcommand{\bais}{\mathbf{b}}%Bais
\newcommand{\act}{\texttt{g}}%Activation
\newcommand{\loss}{L}
\newcommand{\pdata}{\hat{p}_{\texttt{data}}}
\newcommand{\nsize}{n}
\newcommand{\param}{\boldsymbol{\theta}}
\newcommand{\featmap}{\boldsymbol{\phi}}
\newcommand{\EV}{\mathbb{E}}









\title{Deep Learning for Image Analysis - \\
	   Deep Learning for object detection}
\author{Thomas Walter, PhD}
\date{Centre for Computational Biology (CBIO) \\
	  MINES Paris-Tech, PSL Research University \\
	  Institut Curie, PSL Research University \\
	  INSERM U900}


%To include LOGO?
%\logo{\includegraphics[width=.1\columnwidth]{MinesLogo}}
\useinnertheme{rounded}
\usecolortheme{rose}

\usepackage[absolute,overlay]{textpos}

\usepackage{xcolor}
\definecolor{lightblue}{RGB}{0,200,255}


\setbeamertemplate{footline}[frame number]{}
\setbeamertemplate{navigation symbols}{}
\setbeamertemplate{section in toc}[square]
\setbeamertemplate{items}[square]

%% For image credits on image bottom right
\usepackage[absolute,overlay]{textpos}
\setbeamercolor{framesource}{fg=gray}
\setbeamerfont{framesource}{size=\tiny}
\newcommand{\source}[1]{\begin{textblock*}{4cm}(8.7cm,8.6cm)
    \begin{beamercolorbox}[ht=0.5cm,right]{framesource}
      \usebeamerfont{framesource}\usebeamercolor[fg]{framesource} Credits: {#1}
    \end{beamercolorbox}
\end{textblock*}}


\begin{document}

\begin{frame}
\titlepage
\end{frame}

\begin{frame}{Overview}
\tableofcontents
\end{frame}

%%%%%%%%%%%%%%%%%%%%%%%%%%%%%%%%%%%%%%%%%%%%%%%%%%%%%%%%%%%%%%%%%%%%%%%%%
%%%%%%%%%%%%%%%%%%%%%%%%%%%%%%%%%%%%%%%%%%%%%%%%%%%%%%%%%%%%%%%%%%%%%%%%%
\section{Introduction}
\frame{\frametitle{Overview}\tableofcontents[currentsection]}

% \begin{frame}{Definition of Artificial Intelligence and Machine Learning}
% %\begin{definition}
% %	Artificial intelligence (AI) is intelligence demonstrated by machines, in contrast to the natural intelligence displayed by humans and other animals.
% %\end{definition}
% \begin{itemize}
% 	\item The definition of the term \textbf{intelligence} is highly controversial. Usually, one understands by intelligence the capacity of an individual to reason logically, to understand complexity, to learn more or less abstract concepts, to plan and to solve problems in varying conditions.
% 	\item \textbf{Artificial intelligence (AI)} is intelligence demonstrated by machines, in contrast to the natural intelligence displayed by humans and other animals.
% 	\item AI effect: \emph{"AI is whatever hasn't been done yet"}.
% 	\item In 1956, AI became a field of research. AI can be broken down into many subfields: knowledge representation, planning, natural language processing, object manipulation (robotics), machine learning, $\ldots$
% 	\item \textbf{Machine Learning} is concerned with the technology that enables computer programs to improve their performance at a certain task by experience.
% \end{itemize}
% \end{frame}

% \begin{frame}{A short history of artificial intelligence}
% \begin{itemize}
% 	\item The dream to create machines that can think and act has been present in literature and mythology since antiquity (e.g. the myth of \emph{"Talos"} or \emph{"Rossum's Universal Robots"} by Karel \v{C}apek, 1920).
% 	\item AI as a scientific discipline took its beginning with the publication of Alan Turing in 1950 (Turing test). The question \emph{"Can machines think?"} was asked scientifically.
% 	\item Other theoretical bases were developed by Wiener (cybernetics) and Shannon (information theory).
% 	\item The term \emph{"artificial intelligence"} was cornered in 1956, in the famous Dartmouth conference, where AI has been established as a field.
% \end{itemize}
% \end{frame}

% \begin{frame}{A short history of machine learning}
% \begin{figure}[htb]
%   \centering
%   \subfloat{\includegraphics[height=0.23\textheight]{../graphics/Mark_1_perceptron.jpeg}}\hspace{2cm}
%   \subfloat{\includegraphics[height=0.23\textheight]{../graphics/perceptron.jpg}}
%   \caption{The Mark I Perceptron}
% \end{figure}
% \vspace{-.5cm}
% \begin{itemize}
% 	\item The theoretical foundations go back to the early 19th century (e.g. Bayesian theory and Least Squares).
% 	\item Even before machines came into play, there was a lot of interest in finding methods to derive rules from data (\emph{data fitting}). These methods are part of what we call Machine Learning today (linear regression, Bayesian theory, Logistic regression, Linear discriminant analysis, Markov chains).
% 	\item In 1958, Rosenblatt published his \emph{Perceptron} (basically a linear classifier), which is seen as the predecessor of Neural Networks.
% \end{itemize}
% \end{frame}
\begin{frame}{Classification, segmentation and detection}
% \begin{figure}[htb]
%   \centering
%   \subfloat{\includegraphics[height=0.6\textheight]{../graphics/NatureGo.png}}\hspace{2cm}
%   \subfloat{\includegraphics[height=0.6\textheight]{../graphics/NatureSkinLesions.png}}
%   \caption{Two recent success stories of Artificial Intelligence: AI beats the world champion in GO and dermatologists in skin cancer detection.}
% \end{figure}
\begin{itemize}
	\item Image Classification: assign a label to each image. 
	\item Image Segmentation: partition the image, i.e. assign a label to each pixel.
	\item Image understanding requires the estimation of concepts and locations of objects contained in an image.
	\item Problem definition: Object detection is ac omputer vision task that aims to detect instances of semantic objects of certain classes in images \cite{Ren2017, Zhao2019}.
	\item Object detection has thus two components:
	\begin{itemize}
		\item \textit{Object localization:} to determine where objects are located in a given image
		\item \textit{Object classification:} which category each object belongs to
	\end{itemize}
\end{itemize}
%\vspace{-.5cm}
\end{frame}

\begin{frame}{Object detection: example}
\begin{figure}[htb]
   \centering
   \includegraphics[width=0.95\textwidth]{../graphics/Detection_example1.pdf}
   \caption{Object detection in action. Image taken from \cite{Liu2016}}
\end{figure}
\end{frame}

\begin{frame}{Object detection: example}
\begin{figure}[htb]
   \centering
   \includegraphics[width=0.95\textwidth]{../graphics/Detection_example2.pdf}
   \caption{Object detection in action. Image taken from \cite{Liu2016}}
\end{figure}
\end{frame}

\begin{frame}{Object detection: example}
\begin{figure}[htb]
   \centering
   \includegraphics[width=\textwidth]{../graphics/Detection_example3.pdf}
   \caption{Object detection in action. Image taken from \cite{Liu2016}}
\end{figure}
\end{frame}

% \begin{frame}{Difficulties in object detection}
% \begin{itemize}
% \item Orientation: an object can appear under any variation in the image. 
% \end{itemize}
% \end{frame}

\section{Architectures for object detection}
\begin{frame}{Early approaches: the sliding window approach}
\begin{figure}[htb]
   \centering
   \includegraphics[width=0.8\textwidth]{../graphics/mitosis_detection.pdf}
   \caption{Mitosis detection in stained tissue sections \cite{Ciresan2013}.}
\end{figure}

\begin{itemize}
\item Approach was the winner of a mitosis detection challenge \cite{Ciresan2013}. 
\item Fixed size sliding window approach: each crop is presented to a CNN.
\item The posterior probability is stored as an image value.
\item Local maxima of this probability map indicate the presence of an object.
\item Special case of object detection: the size of the objects was known before.
\end{itemize}
\end{frame}

\begin{frame}{A milestone in object detection: R-CNN}
\begin{figure}[htb]
   \centering
   \includegraphics[width=0.8\textwidth]{../graphics/R-CNN.pdf}
   \caption{R-CNN strategy \cite{Girshick2014}}.
\end{figure}
\begin{itemize}
	\item Starting from region proposals (by any method): $\sim 2000$ regions). 
	\item Warping / Cropping of the selected regions into fixed resolution and extraction of a 4096-dimensional feature vector with a pretrained CNN. 
	\item Classification with SVM (object types and background). 
	\item Adjustment by bounding box regression
	\item Filtering with greedy non-maximum suppression (NMS): removal of regions with low overlap with a single object. 
\end{itemize}
%\vspace{-.5cm}
\end{frame}

\begin{frame}{Drawbacks of R-CNN}
\begin{itemize}
\item Fixed input size for the CNN: distortion and rescaling of images is necessary.
\item Multi-stage pipeline.
\item Training is expensive (time and space).
\item Computational expensive, sub-optimal region proposal step.
\end{itemize} 
\end{frame}

\begin{frame}{Fast R-CNN}
\begin{figure}[htb]
   \centering
   \includegraphics[width=0.8\textwidth]{../graphics/Fast_R-CNN.pdf}
   \caption{Fast R-CNN \cite{Girshick2015}}.
\end{figure}
\begin{itemize}
	\item The entire image is processed by a neural network: generation of feature maps.
	\item Region are proposed by some algorithm (as before).
	\item To each region, a ROI pooling layer is applied. 
\end{itemize}
%\vspace{-.5cm}
\end{frame}


\begin{frame}{Fast R-CNN: ROI pooling layer}
\begin{figure}[htb]
   \centering
   \includegraphics[height=0.4\textheight]{../graphics/ROI_pooling_layer.pdf}
   \caption{ROI pooling layer in Fast R-CNN}.
\end{figure}
\begin{itemize}
\item Each region of arbitrary size $w \times h$ is divided into $W \times H$ tiles. $W$ and $H$ are fixed, where as $w$ and $h$ are arbitrary.
\item For each tile, the maximum is calculated (max-pooling operation) in the feature map.
\item The output (fixed size) can then be processed by dense layers. 
\end{itemize}
%\vspace{-.5cm}
\end{frame}

\begin{frame}{Fast R-CNN: Output layer}
% \begin{figure}[htb]
%    \centering
%    \includegraphics[width=0.8\textwidth]{../graphics/Fast_R-CNN.pdf}
%    \caption{Fast R-CNN \cite{Girshick2015}}.
% \end{figure}
\begin{itemize}
\item Two outputs:
\begin{itemize}
	\item Classification output (with a standard softmax layer)
	\item Bounding box regression: prediction of position and extension offsets with respect to the original region proposal. 
\end{itemize}
\item The loss has thus two components: $\loss_{class}$ which is the standard cross-entropy loss and $\loss_{loc}$, the localization loss ($\loss_1$ loss of the offsets with respect to the proposed regions).
% , i.e. for each sample we have a loss:
% \begin{equation*}
% \loss = - \sum_k y_k \log(\hat{y}_k) + \lambda[y \geq 1] \sum_{j \in \{x,y,w,h\}} \loss_1(v^k_j - \hat{v^k}_j)
% \end{equation*}
% where $v$ is the offset relative to a region proposal. 
\item Trick: during training, the batch is constructed from many objects drawn from very few images. 
\item For the prediction, each class gets its own region proposal, that is processed individually with non-maxima suppression. 
\end{itemize}
\end{frame}

% \begin{frame}{Fast R-CNN}
% \begin{itemize}
% \item Fast
% \end{itemize}
% \end{frame}

\begin{frame}{Faster R-CNN: motivation}
\begin{itemize}
	\item Fast R-CNN solves nearly all problem of R-CNN, and is end-to-end given a set of region proposals.
	\item The problem is that we still need to make region proposals to start with (time-consuming and two-stage algorithm).
	\item Faster R-CNN \cite{Ren2017} trains a network called Region Proposal Network (RPN) to overcome this issue.
\end{itemize}
\end{frame}

\begin{frame}{Faster R-CNN: Idea}
\begin{figure}[htb]
   \centering
   \includegraphics[width=0.4\textheight]{../graphics/Faster_R-CNN2.pdf}
   \caption{Faster R-CNN \cite{Ren2017}}.
\end{figure}
The idea is to share convolutional feature maps at test-time, i.e. to use the CNN feature maps calculated for the entire image for both region proposal and object classification \cite{Ren2017}. 
\end{frame}

\begin{frame}{Faster R-CNN: shared layers}
\begin{figure}[htb]
   \centering
   \includegraphics[width=0.8\textwidth]{../graphics/Faster_R-CNN.pdf}
   \caption{Faster R-CNN \cite{Ren2017}}.
\end{figure}
\begin{itemize}
\item First, we run the image through convolutional layers of a CNN and obtain feature maps that will serve both the region proposal and the object classification. 
\item We now slide a small network over the common feature map, that outputs both scores (object yes/no) and regression offsets defining the bounding box. 
\item The size can be relatively small (in \cite{Ren2017}, it is $3 \times 3$); the receptive field is much larger. 
%\item The size of the window needs to be chosen such that the receptive field is larger than all regions in the ground truth. 
\end{itemize}
\end{frame}

\begin{frame}{Faster R-CNN: Region proposal network (RPN)}
\begin{figure}[htb]
   \centering
   \includegraphics[width=0.8\textwidth]{../graphics/Faster_R-CNN.pdf}
   \caption{Faster R-CNN \cite{Ren2017}}.
\end{figure}
\begin{itemize}
\item A Region Proposal Network (RPN) takes an image as input and outputs a set of rectangular object proposals, each with an objectness score.
\item We define $k$ anchor regions (defined by scale and aspect ratio).
\item At each sliding-window location, we simultaneously predict $k$ region proposals from the $k$ anchor regions, for each we predict a score (object yes/no) and the box offsets.
\end{itemize}
\end{frame}

\begin{frame}{Faster R-CNN: Region proposal network (RPN)}
\begin{figure}[htb]
   \centering
   \includegraphics[width=0.8\textwidth]{../graphics/Faster_R-CNN.pdf}
   \caption{Faster R-CNN \cite{Ren2017}}.
\end{figure}
\begin{itemize}
\item We define a combined loss (as for Fast R-CNN), as sum of the classification loss and bounding box regression loss.
\item Classification loss: cross entropy for a binary classifier, indicating whether the region contains an object or not.
\item Regression loss compares for each region proposal its offsets to the anchors with the offsets of the ground truth box.
\end{itemize}
\end{frame}

\begin{frame}{Faster R-CNN: Training}
\begin{figure}[htb]
   \centering
   \includegraphics[width=0.8\textwidth]{../graphics/Faster_R-CNN.pdf}
   \caption{Faster R-CNN \cite{Ren2017}}.
\end{figure}
\begin{itemize}
\item We now simply apply a Fast R-CNN to the region proposals provided by the RPN.
\item The shared layers are trained in an alternating scheme.
\item After this initial training, the shared layers are frozen and the separate layers are trained end-to-end.
\end{itemize}
\end{frame}

\section{Conclusion}

\begin{frame}{Conclusion}
\begin{itemize}
\item Object detection is a major challenge in Computer Vision with applications in biomedical image analysis, autonomous driving, industrial applications, etc.
\item CNNs outperform most traditional methods by a large margin. 
\item Today, object detection is among the most stunning applications of Computer Vision. 
\item There are hundreds of methods, but the most important advances were achieved by R-CNN, Fast R-CNN and Faster R-CNN. 
\item They can be combined with segmentation (Mask R-CNN). 
\end{itemize}
\end{frame}

%The conv-layers are determined by first training the RPN (ImageNet initialization), then we train Fast R-CNN for the proposed regions (again with ImageNet initialization). The shared layers are then frozen.

%\item RPN and classification network thus share the conv-layers. They are specialized later. 

%%%%%%%%%%%%%%%%%%%%%%%%%%%%%%%%%%%%%%%%%%%%%%%%%%%%%%%%%%%%%%%%%%%%%%%%%
%%%%%%%%%%%%%%%%%%%%%%%%%%%%%%%%%%%%%%%%%%%%%%%%%%%%%%%%%%%%%%%%%%%%%%%%%
\section{References}
\begin{frame}[allowframebreaks]
	\frametitle{References}
	\bibliography{object_detection.bib}
\end{frame}


\end{document}
